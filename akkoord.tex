\documentclass{article}

\usepackage{akkoord}

\usepackage[margin=.2in]{geometry}

\usepackage{tabu}

\usepackage{xfrac}

\begin{document}
\akkoord{x,o/-,o/{root},2/1/-,3/3/{root},2/2/-}{D}

\begin{tabu}{ X[1,c,m] | X[5,l,m] | X[8,l,m] }
  \textbf{chord} & \textbf{root 6} & \textbf{root 5} \\ \hline
  \textbf{Maj7 \newline $\triangle$} &
  \akkoord[II]{2/1/1,x,3/3/{$\sharp$7},3/4/3,2/2/5,x}{G\textsubscript{$\triangle$}} &
  \akkoord[II]{x,2/1/1,4/3/5,3/2/{$\sharp$7},4/4/3,x}{C\textsubscript{$\triangle$}} \\ \hline
  \textbf{7} &
  \akkoord[II]{2/1/1,x,2/2/7,3/4/3,2/3/5,x}{G\textsubscript{7}}
  \akkoord{3/2/1,x,3/3/7,1/1/{$\flat$9},3/4/5,x}{G\textsubscript{7$\flat$9}} &
  \akkoord[II]{x,2/1/1,4/3/5,2/1/7,4/4/3,2/1/{(5)}}{C\textsubscript{7}}
  \akkoord{x,3/3/1,2/2/3,3/4/7,1/1/1,x}{C\textsubscript{7}}
  \akkoord{x,3/2/1,2/1/3,3/3/7,3/4/9,3/4/{(5)}}{C\textsubscript{79}}
  \akkoord{x,3/4/1,2/3/3,2/2/6,1/1/1,x}{C\textsubscript{6}}\\ \hline
  \textbf{m7\newline-7} &
  \akkoord[II]{2/2/1,x,2/3/7,2/3/{$\flat$3},2/3/5,x}{G-\textsubscript{7}}
  \akkoord{3/1/1,x,2/2/6,3/3/{$\flat$3},3/4/5,x}{G-\textsubscript{6}} &
  \akkoord[II]{x,2/1/1,4/3/5,2/1/7,3/2/{$\flat$3},2/1/{(5)}}{C-\textsubscript{7}}
  \akkoord{x,3/2/1,1/1/{$\flat$3},3/3/7,4/4/{$\flat$3},x}{C-\textsubscript{7}} \\ \hline
  \textbf{-7$\flat$5 m7$\flat$5 \o \newline halfdim.} &
  \akkoord{3/2/1,x,3/3/7,3/4/{$\flat$3},2/1/{$\flat$5},x}{G\textsubscript{\o}} &
  \akkoord[II]{x,2/1/1,3/3/{$\flat$5},2/2/7,3/4/{$\flat$3},x}{C\textsubscript{\o}} \\ \hline
  \textbf{o dim.} \textit{\tiny all notes root, all dist. 1\sfrac{1}{2}} &
  \akkoord{3/2/1,x,2/1/{$\flat$7},3/3/{$\flat$3},2/1/{$\flat$5},x}{G\textsubscript{o}} &
  \akkoord{x,3/2/1,4/3/{$\flat$5},2/1/{$\flat$7},4/4/{$\flat$3},x}{C\textsubscript{o}} \\ \hline \hline
  chord progression ("trap") &
  \begin{tabu}{X[r] X[l]}
    \multicolumn{2}{c}{scale of G: G A B C D E F$\sharp$} \\
    I & G\textsubscript{$\triangle$} \\
    II & A-\textsubscript{7} \\
    III & B-\textsubscript{7} \\
    IV & C\textsubscript{$\triangle$} \\
    V & D\textsubscript{7} \\
    VI & E-\textsubscript{7} \\
    VII & F$\sharp$\textsubscript{o} \\
  \end{tabu} &
  \begin{tabu}{X[r] X[l]}
    \multicolumn{2}{c}{scale of C: C D E F G A B} \\
    I & C\textsubscript{$\triangle$} \\
    II & D-\textsubscript{7} \\
    III & E-\textsubscript{7} \\
    IV & F\textsubscript{$\triangle$} \\
    V & G\textsubscript{7} \\
    VI & A-\textsubscript{7} \\
    VII & B\textsubscript{o} \\
  \end{tabu} \\ \hline
\end{tabu}


\begin{center}
circle of fifths ("kwintencirkel")

\begin{tikzpicture}
  \draw (0,0) circle [radius=2.5];
  % \filldraw[fill=white] (90:4) circle [radius=.55];
  % \filldraw[fill=white] (60:4) circle [radius=.55];
  % \filldraw[fill=white] (30:4) circle [radius=.55];
  % \filldraw[fill=white] (0:4) circle [radius=.55];
  % \filldraw[fill=white] (-30:4) circle [radius=.55];
  % \filldraw[fill=white] (-60:4) circle [radius=.55];
  % \filldraw[fill=white] (-90:4) circle [radius=.55];
  % \filldraw[fill=white] (-120:4) circle [radius=.55];
  % \filldraw[fill=white] (-150:4) circle [radius=.55];
  % \filldraw[fill=white] (180:4) circle [radius=.55];
  % \filldraw[fill=white] (150:4) circle [radius=.55];
  % \filldraw[fill=white] (120:4) circle [radius=.55];

  \draw (90:3) node {C};
  \draw (60:3) node {G};
  \draw (30:3) node {D};
  \draw (0:3) node {A};
  \draw (-30:3) node {E};
  \draw (-60:3) node {B};
  \draw (-90:3) node {G$\flat$/F$\sharp$};
  \draw (120:3) node {F};
  \draw (150:3) node {B$\flat$};
  \draw (180:3) node {E$\flat$};
  \draw (-150:3) node {A$\flat$};
  \draw (-120:3) node {D$\flat$};
\end{tikzpicture}
\end{center}
\end{document}
